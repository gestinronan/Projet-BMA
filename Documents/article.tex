\documentclass[twocolumn]{article}   		% use "amsart" instead of "article" for AMSLaTeX format

\usepackage[latin1]{inputenc} 
\usepackage[T1]{fontenc}          
\usepackage[francais]{babel}  


\title{Projet Bretagne Mobilit\'e Augment\'e}
\author{\textsc{Gestin Ronan} - \textsc{El-Ouarrat Fatima} - \textsc{Le Floch Guillaume}}
\date{\today}

\begin{document}

\maketitle            					% Create the Title

\newpage		
\clearpage					% Start a new page

%\tableofcontents					% Create the table of contents

\newpage							% Start a new page

\section{Contexte}

	% Contexte du projet
	\subsection{Projet BMA}
	Dans le cadre, d'un cours d'innovation \'a l'ESIR, nous avons r\'efl\'echis \'a ce que serai la mobilit\'e dans les ann\'ee \'a venir. Il en est resortie diff\'erentes \'etudes dont, la combinaison de plusieurs transports. Cela serait un moyen fiable de fid\'eliser les gens \'a l'utilisations des transports en communs. Mais cette invitation au changement ne peut se faire sans garde fou et principalement  la mise en oeuvre d'outils pour aider les gens \'a se retrouver dans cette nouvelle facon de se mouvoir. Dans un m\^eme temps, a eu lieu l'\'expention grandissante du SmartPhone, outils hightech  casimment incontournable de nos jours. Il en est donc ressortit que cet outil est l'un des meilleurs support pour cette l'incitation au changement. Encore plus lorsque l'on connais l'urgence de celle ci.
	% Description du projet BMA
	Le projet BMA regroupe un grand nombre d'acteur de niveau national. Le but du projet est de trouver des moyens de transport moins couteux et moins \'emettrice en gaz \`a effet de serre. 
	Leurs but est de faire \'evolu\'e la facon avec laquelle les gens voyage et ce d\'eplace que ce soit dans leurs milieu professionnel ou personnel.
	
	% Description de cybel
	\subsection{Cybel France}
	
	Cybel assiste les premiers groupes europ\'eens dans la d\'efinition de leur strat\'egie :

		M\'ethode de prospective - Analyse de sc\'enarios
		Portefeuilles d'activit\'es actuels et cibles
		Positionnement concurrentiel
		Analyse de chaines de valeur
	

\section{Background}

% Introduction au serveur
Pour la r\'ealisation de ce projet, nous avons choisi d'utiliser plusieurs technologies. Voici une courte introduction \'a chacune de ces technologies.

	% Presentation NodeJs
	\subsection{NodeJS}

	La partie serveur du projet est r\'ealis\'ee gr\^ace au frameword NodeJs. NodeJs est encore en version Beta mais commence \'a prendre une place non n\'egligeable dans le domaine du web.
	L'int\'er\^et d'utiliser NodeJs vient du fait qu'il utilise le JavaScript. De ce fait, son \'ex\'ecution est donc tr\`es rapide.
	De plus NodeJs poss\`ede la commande NPM qui permet d'installer de nombreuses librairie javascript afin de pouvoir int\'eragir avec le reste du syst\`eme.

	% Presentation Neo4j
	\subsection{Neo4j}

	Afin de pouvoir trouver le plus rapidement possible le chemin par lequel l'utilisateur dois passer, nous avons construit un graphe contenant tous les arr\^ets de transport publique de Rennes. 
	Chaque noeuds correspond donc \'a un arr\^ets de bus, de v\'elo ou de m\'etro et les arr\^etes qui relis chacun de ces noeuds correspond aux temps n\'ecessaire qu'il faut pour aller d'un noeud \'a un autre.
	Gr\^ace \'a une librairies, nous pouvons cr\'eer des scripts en JavaScript afin de pouvoir cr\'ee notre graphe.

	% Presentation Android
	\subsection{Android}
	
	Android est un systm\`eme d'exploitation mobile cr\'e\'e par Google. A l'heure actuelle, c'est le syst\`eme d'exploitation mobile le plus utilis\'e dans le monde. C'est pourquoi nous avons choisi de cr\'eer notre application pour le syt\`eme Android.
	Le syst\`eme Android \'etant open source, nous n'avons pas besoin de license pour d\'evelopper des applications. Les application Android sont developp\'es en Java. 
	
	
	% Presentation HTML5
	\subsection{Html5} 
	
	Le site Web quand \'a lui, b\'en\'eficie des derni\`eres technologie web. Le HTML5 nous permet donc d'int\'egrer la g\'eolocalisation de l'utilisateur via ca connexion internet.
	
	

% Section concernant ce que l'on veut faire dans le projet, la vision que nous en avons	
\section{Approche (Comment le porbleme a \'et\'e trait\'e}

	Les personnes qui cherchent a utiliser les nouveaux moyens de transports, ont besoin d'un outil \'a la fois, rapide, facile d'utilisation et surtout fiable.
	Dans cette d\'emarche, nous avons d/'ecider de proposer un application avec un design  simple, une carte et une option pour poser les question sur les trajet que l'on peut glisser de haut en bas. Pour \'eviter un trop grand nombre d'appel vers les serveurs des diff\'erents partenaires nous avons travaill\'es sur un serveur personnel Nodejs, cela nous permet de pouvoir travaillier \'a la fois sur nos donn\'es brutes et sur les informations de trajet. Pour c'est derni\`eres, nous avons mis en place un graph comportant des d\'ependances pour illustrer les trajets possibles. L'application peut int\'eroger le serveur pour mettre en place les diff\'erents lien possible.



% Section concernant le travaille effectu�
\section{Travail R\'ealis\'e}

% Section concernant le serveur
	\subsection{Le Serveur}
	
	Nous avons choisi pour ce projet d'utiliser le framework NodeJs. NodeJs est un Framework utilisant le language JavaScript. 
	Ce serveur a pour but de recevoir, traiter et renvoyer les donn\'ee aux diff\'errantes interfaces utilisateurs.
	
	
	% Description de la base de donner Mysql
	\subsubsection{La base de donn\'ee MySQL}
	
	Les donn\'ees sont stock\'des dans une base de donn\'ee mysql. La base de donn\'ee contient toutes les informations concernant les arr\^ets de bus, de v\'elo, de m\'etro ainsi que toutes les informations a propos des horaires et des lignes de bus.
	Toutes les donn\'ees concernant les Bus, V\'elos et M\'etro sont en libres service sur le site http://data.keolis-rennes.com/.
	
	La r\'ecup\'eration des donn\'des se fait de diff\'errantes mani\`eres. Les donn\'des relatifs aux arr\^ets de v\'elos et de m\'ethos sont accessibles via une API. Pour r\'ecup\'erer les donn\'des, nous faisons un appel AJAX depuis le serveur puis le r\'esultat est ajout\'e dans la Base de donn\'e MySql.
	En ce qui concerne les arr\^ets de bus, ces donn\'des sont des fichiers statiques que nous devons t\'el\'echarg\'ee. Une fois ces fichiers t\'el\'echarg\'es, nous les avons pars\'es puis ajout\'e les donn\'ees dans la base de donn\'ee.
	L'int\'eraction entre la base de donn\'ee MySql et le serveur NodeJs se fait gr\^ace \'a la librairie 'node-mysql'.
	
	% Description de la base de donner de graph
	\subsubsection{La base de donn\'ee de Graph neo4j }
	
	Afin de pouvoir g\'en\'erer les itin\'eraires utilisant les diff\'erents moyens de transport disponible de Rennes, nous avons d\'ecid\'e de cr\'eer un graphe qui comprend tous les arr\^ets de bus, v\'elos et m\'etro.
	Pour cr\'eer ce graphe, nous avons utilis\'e la base de donn\'ee Neo4j. Cette base de donn\'e permet de cr\'eer et d'enregistrer les noeuds ainsi que les relations entre les noeuds.
	
	L'int\'eraction entre la base de donn\'ee Neo4j et le serveur NodeJs se fait gr\^ace \'a la librairie 'node-neo4j'.  

	% Section concernant l'application android
	\subsection{L'aplication Android}
	
	Pour ce projet nous somme partie sur une application android, pour sa simplicit\'e d'approche. Mais aussi car c'est le syst\`eme d'exploitation mobile le plus r\'ependu au monde a l'heur actuel.
	% Description de la vue principal
	\subsubsection{Une acitvit\'e unique}
	
	Pour une \'ergonomie unique nous avons d\'ecider de mettre en place une acitivit\'e unique avec l'affichage de la carte directement pour l'utilisateur. avec certaine donne de mobilit\'e pr\'e affichier, tel que les v\'elo, train, metro... avec la possibilit\'e dans le menu  de pr\'ef\'eance d'ajouter les bus ou de r\'egl\'e le niveau de l'utilisateur en v\'elo..
	
	Pour les la partie de gestion des route nous avons mis en place ce que l'on appel un slidingdrawer ( une vu qui ce glisse sur la carte). Elle a pour but de r\'ecuperer les information de l'utilisateur pour sont trajet. Elle proposera une aide \'a la compl\'etion des information ( nom de arret) et le choix des diff\'erent moyen de transport pour les trajet.
 	
	% Description des appel au serveur
	\subsubsection{Les appels au serveur}
	
	Dans ce projet base sur une communicaiton client serveur,  l'une des principal difficult\'e \'etait de faire le lien entre c'est deux entit\'ees. Pour cela il a \'et\'e d\'ecider de mettre en place des tache asynchrone pour permettre a l'application de vivre et de faire en m\^eme temps la mise a jour des diff\'erente donn\'ee pour l'utilisateur. Nous avons utiliser pour cela les Intentservices qui permette de communique des donn\'des via les intent  entre les taches asynchrone et les vu ici la vu principal est la map.
	
	% Section concernant le site web
	\subsection{Le Site Web}
	Le site permet de chercher des itin\'eraires de bien diverses facons: l'utilisateur peut entrer manuellement l'adresse  de d\'epart et d'arriv\'ee (et \'eventuellement ajouter jusqu'\'a 3 \'etapes) dans les champs pr\'evus \'a cet effet puis cliquer sur le bouton 'Rechercher'.
Encore plus simple d'utilisation, la d\'efinition des points sur la carte permet aux utilisateurs de d\'efinir en quelques clics de souris l'itin\'eraire qu'ils souhaitent calculer. Un Clique sur les lieux que l'utilisateur souhaite inclure \'a son itin\'eraire. Apr\`es 5 points d\'efinis, les points suivants viennent s'ajouter sur les champs r\'eserver pour le d\'epart et l'arriver. Comme il peut aussi rechercher par mots on \'ecrivant un mot...
Si l'utilisateur est \'a pied ou avec son v\'elo, le site lui propose le trajet de son point de d\'epart \'a son point d'arriv\'ee, en passant par les \'etapes qu'il a d\'efinie lors de sa recherche.
Les itin\'eraires sont calcul\'es \'a partir des informations envoy\'ees par le serveur. 
Si l'utilisateur utilise un transport public K\'eolis (V\'eloStar, bus, m\'etro), la r\'eponse fourni par le serveur lui permet d'avoir la station la plus proche de son points d\'efinis et lui indique ensuite comment faire pour rejoindre les autres points qu'il a d\'efini.
Sur la carte on a d\'efini un petit menu qui permis aux utilisateurs de voir directement sur la carte les stations de m\'etro, v\'elo, borne \'electrique en s\'electionnant l'une ou plusieurs options.  
Le site a \'et\'e r\'ealis\'e en HTML5, CSS et JavaScript



% Section concernant la validation du travail par l'entreprise
\section{Validation du proff\'essionnel}
Lors des r\'eunions avec le PDG, ce dernier \'etait satisfait de notre travail r\'ealiser lors de chaque r\'eunion, il nous a confi\'e de rajouter des taches sur le projet comme l'ajout des bornes \'electrique.Il a souhait\'e aussi qu'on puisse proposer aux clients des services, en l'int\'egrant dans l'application en utilisant prolog.


% Section pour la conclusion
\section{Conclusion}

Ce projet nous a permis de pouvoir mettre en application les diff\'erentes comp\'etence de chacun. Nous avons ainsi pu cr\'eer une base de donn\'ee MySql pour stocker toutes nos donn\'ees concernant les diff\'erents moyens et types de transport publique disponible dans la r\'egion Rennaise. Nous avons aussi cr\'e\'es une base de donn\'ee de graphe pour \^etre de capable de g\'enerer des it\'eraires dans la ville de Rennes. Un serveur en NodeJs a \'et\'e cr\'e\'e dans le but de pouvoir g\'erer les requ\^etes venant \'a la fois de notre site web ainsi que de notre application internet. Le serveur fait les query vers les diff\'errantes base de donn\'ees et renvoie les informations \'a l'application ou au site web.
Gr\^ace \'a ce projet, nous avons pu apprendre \'a travailler \'a plusieurs sur un m\^eme sujet. Afin de bien g\'erer ce projet, nous avons du apprendre \'a respecter un emploi du temps et \'a rendre nos travaux en temps et heure.
De plus le suivi de projet de la part de Mr Chevalier nous a permis d'avoir une id\'ee de comment comprendre et retranscrire les id\'es abord\'ees lors de nos diff\'errantes r\'union dans notre travail. 
Ce projet a pour nous \'et\'e une tr\`es bonne exp\'erience.
De plus, nous tenons \'a remercier Mr Chevalier ainsi que Mr Bourcier pour l'aide qu'ils nous ont fournis durant tout ce projet.


% Section concernant la bibliographie
\section{Bibliographis}

\begin{flushleft}
Projet BMA: http://bretagne-mobilite-augmentee.fr/
\newline 
Cybel: www.cybel.fr
\newline 
Le code du projet: https://github.com/gestinronan/Projet-BMA
\newline 
La documentation du projet: https://github.com/gestinronan/Projet-BMA/wiki/
\newline 
NodeJs: http://nodejs.org/
\newline 
Android: http://developer.android.com/index.html
\newline 
Html5: http://www.html5rocks.com/fr/
\end{flushleft}



	
\end{document}