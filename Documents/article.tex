\documentclass[twocolumn]{article}   		% use "amsart" instead of "article" for AMSLaTeX format

\usepackage[latin1]{inputenc} 
\usepackage[T1]{fontenc}          
\usepackage[francais]{babel}  
\usepackage{graphicx}					% Package pour inclure graphique
\usepackage{graphics}

\title{Projet Bretagne Mobilit\'e Augment\'ee}
\author{\textsc{Gestin Ronan} - \textsc{El-Ouarrat Fatima} - \textsc{Le Floch Guillaume}}
\date{\today}

\begin{document}

\maketitle            					% Create the Title

\newpage		
\clearpage						% Start a new page

%\tableofcontents					% Create the table of contents

\newpage							% Start a new page

\section{Contexte}

	% Contexte du projet
	\subsection{Projet BMA}
	Dans le cadre d'un cours d'innovation \`a l'ESIR, nous avons r\'efl\'echi \`a ce que serait la mobilit\'e dans les ann\'ees \`a venir. Il en est ressorti diff\'erentes \'etudes dont la combinaison de plusieurs transports. Cela serait un moyen fiable de fid\'eliser les gens \`a l'utilisation des transports en commun. Mais cette invitation au changement ne peut se faire sans garde fou et principalement la mise en oeuvre d'outils pour aider les gens \`a se retrouver dans cette nouvelle fa\c con de se mouvoir. Dans un m\^eme temps, a eu lieu l'expansion grandissante du Smartphone, outil hightech quasiment incontournable de nos jours. Il en est donc ressorti que cet outil est l'un des meilleurs supports pour cette incitation au changement. Surtout lorsque l'on conna\^it l'urgence de celle ci. Le projet BMA regroupe un grand nombre d'acteurs de niveau national. Le but du projet est de trouver des moyens de transport moins couteux et moins \'emetteurs en gaz \`a effet de serre. Leur but est de faire \'evoluer la fa\c con dont les gens voyagent et se d\'eplacent que se soit dans leur milieu professionnel ou personnel.	
	% Description de cybel
	\subsection{Cybel France}
	Cybel assiste les premiers groupes europ\'eens dans la d\'efinition de leur strat\'egie :
	
	\begin{itemize}
		\renewcommand{\labelitemi}{$\bullet$}
		\item M\'ethode de prospective - Analyse de sc\'enarios.
		\item Portefeuilles d'activit\'es actuels et cibles.
		\item Positionnement concurrentiel.
		\item Analyse de chaines de valeurs.
	\end{itemize}	

\section{Background}

% Introduction au serveur
Pour la r\'ealisation de ce projet, nous avons choisi d'utiliser plusieurs technologies. Voici une courte introduction \`a chacune de ces technologies.
	% Presentation NodeJs
	\subsection{NodeJS}

	La partie serveur du projet est r\'ealis\'ee gr\^ace au framework NodeJs. NodeJs est encore en version Beta mais commence \`a prendre une place non n\'egligeable dans le domaine du web. L'int\'er\^et d'utiliser NodeJs vient du fait qu'il utilise le JavaScript. De ce fait, son ex\'ecution est donc tr\`es rapide. De plus NodeJs poss\`ede un manager de paquet appel\'e NPM qui permet d'installer de nombreuses librairies javascript afin de pouvoir interagir avec le reste du syst\`eme.

	% Presentation Neo4j
	\subsection{Neo4j}

	Afin de pouvoir trouver le plus rapidement possible le chemin par lequel l'utilisateur doit passer, nous avons construit un graph contenant tous les arr\^ets de transports publics de Rennes. Chaque noeud correspond donc \`a un arr\^et de bus, de v\'elo ou de m\'etro et les arr\^etes qui relient chacun de ces noeuds correspondent au temps n\'ecessaire qu'il faut pour aller d'un noeud \`a un autre. Gr\^ace \`a une librairie, nous sommes capable de g\'en\'erer notre graphe \`a partir de scripts JavaScript.
	
	% Presentation Android
	\subsection{Android}
	
	Android est un syst\`eme d'exploitation mobile cr\'e\'e par Google. A l'heure actuelle, c'est le syst\`eme d'exploitation mobile le plus utilis\'e dans le monde. C'est pourquoi nous avons choisi de cr\'eer notre application pour le syst\`eme Android. Le syst\`eme Android \'etant open source, nous n'avons pas besoin de licence pour d\'evelopper des applications. Les applications Android sont d\'evelopp\'ees en Java.
	
	% Presentation HTML5
	\subsection{Html5} 
	
	Le site Web quand \`a lui, b\'en\'eficie des derni\`eres technologies web. Le HTML5 nous permet donc d'int\'egrer la g\'eolocalisation de l'utilisateur via sa connexion internet.
	

% Section concernant ce que l'on veut faire dans le projet, la vision que nous en avons	
\section{Approche}

	Les personnes qui cherchent \`a utiliser les nouveaux moyens de transport ont besoin d'un outil \`a la fois rapide, facile d'utilisation et fiable. Dans cette d\'emarche, nous avons d\'ecid\'e de proposer une application avec un design simple (une carte ainsi qu'un menu d\'eroulant permettant \`a l'utilisateur de pouvoir s\'electionner son point de d\'epart et d'arriv\'ee). Pour \'eviter un trop grand nombre d'appels vers les serveurs des diff\'erents partenaires nous avons travaill\'e sur un serveur personnel (en Nodejs), cela nous permet de pouvoir travailler \`a la fois sur nos donn\'ees brutes et sur les informations de trajet. Pour ces derni\`eres, nous avons mis en place un graph comportant des d\'ependances pour illustrer les diff\'erents trajets possibles. L'application peut interroger le serveur pour mettre en place les diff\'erents liens possibles.
	 En plus de l'application Android, l'utilisateur peut aussi utiliser le site web pour s\'electionner et visualiser ses itin\'eraires. Le site web a \'et\'e realis\'e de la m\^eme fa\c con que l'application, c'est \`a dire un design et une utilisation simple.
	
	\begin{figure}[!h]
	\centering
   	\includegraphics[scale=0.4]{schema.jpg}
   	\caption{\label{bookOcl} Architecture projet}
	\end{figure}
Comme le montre la figure 1, nous avons d\'ecid\'e d'orienter le projet vers une architecture client- serveur. Nous avons fait ce choix car, avec ce type d'architecture, nous pouvons facilement multiplier le nombre d'interface utilisateur.

% Section concernant le travaille effectu\'e
\section{Travail R\'ealis\'e}

% Section concernant le serveur
	\subsection{Le Serveur}
	
	Nous avons choisi pour ce projet d'utiliser le framework NodeJs. NodeJs est un Framework utilisant le langage JavaScript. Ce serveur a pour but de recevoir, traiter et renvoyer les donn\'ees aux diff\'erentes interfaces utilisateurs.
	
	
	% Description de la base de donner Mysql
	\subsubsection{La base de donn\'ees MySQL}
	
	Les donn\'ees sont stock\'ees dans une base de donn\'ees mysql. La base de donn\'ees contient toutes les informations concernant les arr\^ets de bus, de v\'elo, de m\'etro ainsi que toutes les informations \`a propos des horaires et des lignes de bus. Toutes les donn\'ees concernant les Bus, V\'elos et M\'etros sont en libre service sur le site http ://data.keolis-rennes.com/.
La r\'ecup\'eration des donn\'ees se fait de diff\'erentes mani\`eres. Les donn\'ees relatives aux arr\^ets de v\'elos et de m\'etros sont accessibles via une API. Pour r\'ecup\'erer ces donn\'ees, nous faisons un appel AJAX depuis le serveur puis le r\'esultat est ajout\'e dans la Base de donn\'ees MySql. En ce qui concerne les arr\^ets de bus, ces donn\'ees sont des fichiers statiques que nous devons t\'el\'echarger. Une fois ces fichiers t\'el\'echarg\'es, nous les parsons, et les donn\'ees sont directement ajout\'ees \`a la dans les tables correspondantes. L'interaction entre la base de donn\'ees MySql et le serveur NodeJs se fait gr\^ace \`a la librairie 'node-mysql'.
	
	% Description de la base de donner de graph
	\subsubsection{La base de donn\'ees de Graph neo4j }
	
	Afin de pouvoir g\'en\'erer les itin\'eraires utilisant les diff\'erents moyens de transport disponibles de Rennes, nous avons d\'ecid\'e de cr\'eer un graphe qui comprend tous les arr\^ets de bus, v\'elo et m\'etro. Pour cr\'eer ce graphe, nous avons utilis\'e la base de donn\'ees Neo4j. Cette base de donn\'ees permet de cr\'eer et d'enregistrer les noeuds ainsi que les relations entre les noeuds.
L'interaction entre la base de donn\'ees Neo4j et le serveur NodeJs se fait gr\^ace \`a la librairie 'node-neo4j'. Afin de g\'en\'erer les arr\^etes repr\'esentant les lignes de bus, nous avons utilis\'e les donn\'ees pr\'esentes dans la base de donn\'ees MySql. En ce qui concerne les arr\^etes repr\'esentant les distances \`a v\'elo et \`a pied, nous avons utilis\'e l'API CloudMade qui gr\^ace \`a un appel HTTP de type POST nous renvoie le temps n\'ecessaire pour aller d'un point \`a un autre en v\'elo ou \`a pied.

	% Section concernant l'application android
	\subsection{L'application Android}
	
	Afin de rendre notre application disponible sur un syst\`eme mobile, nous avons choisi de le faire sur une base Android.
		
	\begin{figure}[!h]
	\centering
   	\includegraphics[scale=0.3]{archiapp.jpg}
   	\caption{\label{bookOcl} Architecture application}
	\end{figure}
	
	La figure 2 repr\'esente la fa\c con dont l'application Android a \'et\'e r\'ealis\'ee. Comme vous pouvez le voir, au moment du lancement, deux fonctions sont lanc\'ees, tout d'abord, un appel HTTP est effectu\'e vers le serveur afin de r\'ecup\'erer les informations. Puis la carte s'affiche.
	
	% Description de la vue principal
	\subsubsection{Une acitvit\'e unique}
	
	Pour une ergonomie unique nous avons d\'ecid\'e de mettre en place une activit\'e unique avec l'affichage direct de la carte. Lors de l'affichage de la carte, certaines donn\'ees de mobilit\'e sont pr\'e-affich\'ees, telles que les arr\^ets de v\'elo, de train et de m\'etro. Cependant, l'utilisateur a la possibilit\'e dans le menu de pr\'ef\'erence d'ajouter l'affichage des arr\^ets de bus.
Pour la partie de gestion des routes, nous avons mis en place ce que l'on appel un slidingdrawer (une vue qui se glisse sur la carte). Elle a pour but de r\'ecup\'erer les informations de l'utilisateur pour son trajet. Elle proposera une aide \`a la compl\'etion des informations (nom de l'arr\^et) et le choix des diff\'erents moyens de transport pour les trajets.
 	
	% Description des appel au serveur
	\subsubsection{Les appels au serveur}
	
	Dans ce projet bas\'e sur une communication client serveur, l'une des principales difficult\'es \`a \'et\'e de faire le lien entre ces deux entit\'es. Pour cela il a \'et\'e d\'ecid\'e de mettre en place des taches asynchrones pour permettre \`a l'application de vivre et de faire en m\^eme temps la mise \`a jour des diff\'erentes donn\'ees pour l'utilisateur. Nous avons utilis\'e pour cela les Intent-services qui permettent de communiquer des donn\'ees via les intents entre les taches asynchrones et les vues (dans notre application, la vue principale est la map).
	
	% Section concernant le site web
	\subsection{Le Site Web}
	Le site permet de chercher des itin\'eraires de bien diverses fa\c cons : l'utilisateur peut entrer manuellement l'adresse de d\'epart et d'arriv\'ee (et \'eventuellement ajouter jusqu'\`a 3 \'etapes) dans les champs pr\'evus \`a cet effet puis cliquer sur le bouton 'Rechercher'. Encore plus simple d'utilisation, la d\'efinition des points sur la carte permet aux utilisateurs de d\'efinir en quelques clics de souris l'itin\'eraire qu'ils souhaitent calculer. Un Clique sur les lieux que l'utilisateur souhaite inclure \`a son itin\'eraire. Apr\`es 5 points d\'efinis, les points suivants viennent s'ajouter sur les champs r\'eserv\'es pour le d\'epart et l'arriv\'ee. Comme il peut aussi rechercher par mots en \'ecrivant un mot... Si l'utilisateur est \`a pied ou avec son v\'elo, le site lui propose le trajet de son point de d\'epart \`a son point d'arriv\'ee, en passant par les \'etapes qu'il a d\'efini lors de sa recherche. Les itin\'eraires sont calcul\'es \`a partir des informations envoy\'ees par le serveur. Si l'utilisateur utilise un transport public K\'eolis (V\'eloStar, bus, m\'etro), la r\'eponse fourni par le serveur lui permet d'avoir la station la plus proche de son point d\'efini et lui indique ensuite comment faire pour rejoindre les autres points qu'il a d\'efini. Sur la carte on a d\'efini un petit menu qui permet aux utilisateurs de voir directement sur la carte les stations de m\'etro, v\'elo, borne \'electrique en s\'electionnant une ou plusieurs options. Le site a \'et\'e r\'ealis\'e en HTML5, CSS et JavaScript.

% Section concernant la validation du travail par l'entreprise
\section{Validation du professionnel}
Lors des r\'eunions avec le PDG, ce dernier \'etait satisfait de notre travail r\'ealis\'e lors de chaque r\'eunion. Il nous a confi\'e de rajouter des taches sur le projet comme l'ajout des bornes \'electriques. Il a aussi souhait\'e que l'on puisse proposer aux utilisateurs des services, ainsi qu'une int\'egration dans l'application en utilisant prolog.

% Section pour la conclusion
\section{Conclusion}

	Ce projet nous a permis de pouvoir mettre en application les diff\'erentes comp\'etences de chacun. Nous avons ainsi pu cr\'eer une base de donn\'ees MySql pour stocker toutes nos donn\'ees concernant les diff\'erents moyens et types de transports publics disponibles dans la r\'egion Rennaise. Nous avons aussi cr\'e\'e une base de donn\'ees de graphes pour \^etre   capable de g\'en\'erer des itin\'eraires dans la ville de Rennes. Un serveur en NodeJs a \'et\'e cr\'e\'e dans le but de pouvoir g\'erer les requ\^etes venant \`a la fois de notre site web ainsi que de notre application Android. Le serveur fait les query vers les diff\'erentes bases de donn\'ees et renvoie les informations \`a l'application ou au site web. Gr\^ace \`a ce projet, nous avons pu apprendre \`a travailler \`a plusieurs sur un m\^eme sujet. Afin de bien g\'erer ce projet, nous avons du apprendre \`a respecter un emploi du temps et \`a rendre nos travaux en temps et heure. De plus le suivi de projet de la part de Mr Chevalier nous a permis d'avoir une id\'ee de comment interpr\'eter les id\'ees abord\'ees lors de nos diff\'erentes r\'eunions dans notre travail. Ce projet a pour nous \'et\'e une tr\`es bonne exp\'erience. De plus, nous tenons \`a remercier Mr Chevalier ainsi que Mr Bourcier pour l'aide qu'ils nous ont fourni durant tout ce projet.
% Section concernant la bibliographie
\section{Bibliographie}

\begin{flushleft}
Projet BMA: http://bretagne-mobilite-augmentee.fr/
\newline 
Cybel: www.cybel.fr
\newline 
Le code du projet: https://github.com/gestinronan/Projet-BMA
\newline 
La documentation du projet: https://github.com/gestinronan/Projet-BMA/wiki/
\newline 
NodeJs: http://nodejs.org/
\newline 
Android: http://developer.android.com/index.html
\newline 
Html5: http://www.html5rocks.com/fr/
\end{flushleft}



	
\end{document}