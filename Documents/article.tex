\documentclass{article}   		% use "amsart" instead of "article" for AMSLaTeX format

\usepackage[latin1]{inputenc} 
\usepackage[T1]{fontenc}          
\usepackage[francais]{babel}  


\title{Projet Bretagne Mobilit\'e Augment\'e}
\author{\textsc{Gestin Ronan} - \textsc{El-Ouarrat Fatima} - \textsc{Le Floch Guillaume}}
\date{\today}

\begin{document}

\maketitle            					% Create the Title
\line(1, 0){350}
\newpage							% Start a new page

\tableofcontents					% Create the table of contents

\newpage							% Start a new page

\section{Contexte}

	% Contexte du projet
	\subsection{Projet BMA}
	
	% Description du projet BMA
	Le projet BMA regroupe un grand nombre d'acteur de niveau national. Le but du projet est de trouver un des moyens de transport moins couteux et moins \'emettrice en gaz \`a effet de serre. 
	
	% Description de cybel
	\subsection{Cybel France}
	
\clearpage
\section{Background}

% Introduction au serveur
Pour la r\'ealisation de ce projet, nous avons choisi d'utiliser plusieurs technologies. Voici une courte introduction \'a chacune de ces technologies.

	% Presentation NodeJs
	\subsection{NodeJS}

	La partie serveur du projet est r\'ealis\'ee gr\^ace au frameword NodeJs. NodeJs est encore en version Beta mais commence \'a prendre une place non n\'egligeable dans le domaine du web.
	L'int\'er\^et d'utiliser NodeJs vient du fait qu'il utilise le JavaScript. De ce fait, son \'ex\'ecution est donc tr\`es rapide.
	De plus NodeJs poss\`ede la commande NPM qui permet d'installer de nombreuses librairie javascript afin de pouvoir int\'eragir avec le reste du syst\`eme.

	% Presentation Neo4j
	\subsection{Neo4j}

	Afin de pouvoir trouver le plus rapidement possible le chemin par lequel l'utilisateur dois passer, nous avons construit un graphe contenant tous les arr\^ets de transport publique de Rennes. 
	Chaque noeuds correspond donc \'a un arr\^ets de bus, de v\'elo ou de m\'etro et les arr\^etes qui relis chacun de ces noeuds correspond aux temps n\'ecessaire qu'il faut pour aller d'un noeud \'a un autre.
	Gr\^ace \'a une librairies, nous pouvons cr\'eer des scripts en JavaScript afin de pouvoir cr\'ee notre graphe.

	% Presentation Android
	\subsection{Android}
	
	Android est un systm\`eme d'exploitation mobile cr\'e\'e par Google. A l'heure actuelle, c'est le syst\`eme d'exploitation mobile le plus utilis\'e dans le monde. C'est pourquoi nous avons choisi de cr\'eer notre application pour le syt\`eme Android.
	Le syst\`eme Android \'etant open source, nous n'avons pas besoin de license pour d\'evelopper des applications. Les application Android sont developp\'es en Java. 
	
	
	% Presentation HTML5
	\subsection{Html5} 
	
	Le site Web quand \'a lui, b\'en\'eficie des derni\`eres technologie web. Le HTML5 nous permet donc d'int\'egrer la g\'eolocalisation de l'utilisateur via ca connexion internet.
	
	
\clearpage
% Section concernant ce que l'on veut faire dans le projet, la visio que nous en avons	
\section{Approche (Comment le porbl`eme a \'et\'e trait\'e}

	
	

\newpage
% Section concernant le travaille effectu�
\section{Travail R\'ealis\'e}

% Section concernant le serveur
	\subsection{Le Serveur}
	
	Nous avons choisi pour ce projet d'utiliser le framework NodeJs. NodeJs est un Framework utilisant le language JavaScript. 
	Ce serveur a pour but de recevoir, traiter et renvoyer les donn\'ee aux diff\'errantes interfaces utilisateurs.
	
	
	% Description de la base de donner Mysql
	\subsubsection{La base de donn\'ee MySQL}
	
	Les donn\'ees sont stock\'des dans une base de donn\'ee mysql. La base de donn\'ee contient toutes les informations concernant les arr\^ets de bus, de v\'elo, de m\'etro ainsi que toutes les informations a propos des horaires et des lignes de bus.
	Toutes les donn\'ees concernant les Bus, V\'elos et M\'etro sont en libres service sur le site http://data.keolis-rennes.com/.
	
	La r\'ecup\'eration des donn\'des se fait de diff\'errantes mani\`eres. Les donn\'des relatifs aux arr\^ets de v\'elos et de m\'ethos sont accessibles via une API. Pour r\'ecup\'erer les donn\'des, nous faisons un appel AJAX depuis le serveur puis le r\'esultat est ajout\'e dans la Base de donn\'e MySql.
	En ce qui concerne les arr\^ets de bus, ces donn\'des sont des fichiers statiques que nous devons t\'el\'echarg\'ee. Une fois ces fichiers t\'el\'echarg\'es, nous les avons pars\'es puis ajout\'e les donn\'ees dans la base de donn\'ee.
	L'int\'eraction entre la base de donn\'ee MySql et le serveur NodeJs se fait gr\^ace \'a la librairie 'node-mysql'.
	
	% Description de la base de donner de graph
	\subsubsection{La base de donn\'ee de Graph neo4j }
	
	Afin de pouvoir g\'en\'erer les itin\'eraires utilisant les diff\'erents moyens de transport disponible de Rennes, nous avons d\'ecid\'e de cr\'eer un graphe qui comprend tous les arr\^ets de bus, v\'elos et m\'etro.
	Pour cr\'eer ce graphe, nous avons utilis\'e la base de donn\'ee Neo4j. Cette base de donn\'e permet de cr\'eer et d'enregistrer les noeuds ainsi que les relations entre les noeuds.
	
	L'int\'eraction entre la base de donn\'ee Neo4j et le serveur NodeJs se fait gr\^ace \'a la librairie 'node-neo4j'.  
	
	% Section concernant l'application android
	\subsection{L'aplication Android}
	
	% Section concernant le site web
	\subsection{Le Site Web}


\newpage
% Section concernant la validation du travail par l'entreprise
\section{Validation du proff\'essionnel}


\newpage
% Section pour la conclusion
\section{Conclusion}



\newpage % Start a new page for the bibio
% Section concernant la bibliographie
\section{Bibliographie}
	
	
\end{document}